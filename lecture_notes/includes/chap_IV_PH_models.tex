% !TEX root = ../main_lecture_notes.tex
\chapter{Modèles à hasard proportionel}\label{chap:Ph_models}
Nous étudions des données $t_1,\ldots, t_n$ censurés à droite 
$$
(x_i,\delta_i) = \left(t_i\land c_i,\ind_{t_i\leq c_i}\right),\text{  }i=1,\ldots, n.
$$
Chaque évènement $i$ est associé à des caractéristiques ou variable explicatives 
$$
z_{i1},\ldots, z_{ip}\text{ }i=1,\ldots, n.
$$
L'objectif est d'étudier la loi conditionelle de la durée $T$ sachant $Z$.
\section{Définition du modèle}
Le modèle à hasard proportionel spécifie la fonction de hasard par 
\begin{equation}\label{eq:cox_proportional_hazard}
h(t; z, \beta) = h_0(t)\e^{z \beta}=h_0(t)\e^{z_1\beta_1+\ldots+z_p\beta_p},
\end{equation}
où la fonction $h_0$ est le risque de base, modifié par les caractéristiques individuelles suivant les coefficients de regressions $\beta = (\beta_1\text{ \ldots }\beta_p)$. L'impact multiplicatif des variables sur le taux de hasard explique le nom du modèle.
\begin{ex}
	\begin{enumerate}
		\item Prenons l'exemple d'un modèle ne comprenant qu'une variable explicative $Z\in\{0,1\}$ alors le risque instantané s'écrit 
		$$
		h(t)=\begin{cases}
		h_0(t),&\text{ si }Z_0 = 0\\
		h_0(t)\e^{\beta},&\text{ sinon}.
		\end{cases}
		$$
		Le risque est multiplié par $\e^\beta$ lorsque $Z_1=1$. Le signe de $\beta$ indique si le risque augmente $\beta>0$ ou diminue $\beta>0$
		\item pour une variable continue $Z\in \RL$, le coefficient de regression donne l'impact d'une variation d'amplitude $1$. En effet, 
		$$
\frac{h_0(t)\e^{\beta\cdot (z+1)}}{h_0(t)\e^{\beta\cdot z}} = \e^{\beta}.
		$$
	\end{enumerate}
\end{ex}
\section{Inférence des paramètres}
La vraisemblance du modèle pour les données $\Data = (x_k,\delta_k)_{k=1,\ldots, n}$ s'écrit 
\begin{equation}\label{eq:likelihood_cox}
\mathcal{L}(\Data;\theta) =\prod_{i=1}^n \left[h_0(x_i)\e^{z_i\cdot \beta}\right]^{\delta_i}\cdot\exp\left(-\e^{z_i\cdot\beta} \int_0^{x_i}h_0(s)\text{d}s\right) ,
\end{equation}
où $\theta = (h_0,\beta)$. Il n'est pas possible d'inférer les paramètres avec la vraisemblance \eqref{eq:likelihood_cox} sans donner une structure au risque de base. 
\begin{remark}
Le modèle \eqref{eq:cox_proportional_hazard} est le modèle de \citet{Cox1972}. L'étude se focalise plus sur l'impact des covariable $Z_1,\ldots, Z_p$ que sur le risque de base $h_0$ qui peut prendre une forme arbitraire. \citet{Cox1972} défini une vraisemblance partielle qui correspond à la probabilité que les évènements se produisent dans l'ordre observé. Nous souhaitons également estimé le risque de base, nous allons donc supposer que $h_0$ est une fonction continue par morceaux, voir \citet{Breslow1975}.
\end{remark}
Soit 
$$
0=v_0<v_1<v_2<\ldots<v_K<v_{K+1}=\infty,
$$
tels que $v_1,\ldots, v_K$ correspondent aux observations ordonnées, distinctes et non censuré. Le risque de base est donnée par
$$
h_0(t)=\sum_{k=1}^{K}\alpha_k\ind_{(v_{k-1}, v_{k}]}(t),t\geq 0.
$$
Les paramètres du modèle sont données par $\theta = (\alpha, \beta)$. La log vraisemblance s'écrit
\begin{eqnarray*}
l(\Data;\theta)&=&\sum_{i=1}^n\delta_i\{\log[h_0(x_i)]+z_i\beta\}-\e^{z_i\beta}\int_0^{x_i}h_0(s)\text{d}s\\
&=& \sum_{i=1}^n\delta_i\sum_{k=1}^K\left[\log(\alpha_k)+\delta_i z_i\beta\right]\ind_{x_i = v_k} -\e^{z_i\beta}\sum_{k=1}^K\alpha_k\int_0^{x_i}\ind_{(v_{k-1}, v_k]}(s)\text{d}s\\
&=& \sum_{k=1}^K\log(\alpha_k)d(v_k)+s_k\beta -\alpha_k\sum_{i=1}^{n}\int_0^{x_i}\e^{z_i\beta}\ind_{(v_{k-1}, v_k]}(s)\text{d}s\\
&=& \sum_{k=1}^K\log(\alpha_k)d(v_k)+s_k\beta -\alpha_k\left\{\sum_{i=1}^{n}\e^{z_i\beta}\left[(v_k - v_{k-1})\ind_{x_i\geq v_k} +(x_i - v_{k-1})\ind_{(v_{k-1},v_k]}(x_i)\right]\right\}\\
&=& \sum_{k=1}^K\log(\alpha_k)d(v_k)+s_k\beta -\alpha_kE_k(\beta),\\
\end{eqnarray*}
où $d(v_k)$ est le nombre d'évènements se produisant exactement à l'instant $v_k$ et $s_k$ la somme des caractéristiques ($\sum_i \delta_i z_i$) de ces évènements. L'estimateur du maximum de vraisemblance $\widehat{\theta}$ est obtenu en annulant les dérivées premières, il vient 
$$
\widetilde{\alpha}_k = \frac{d(v_k)}{E_k(\beta)}, 
$$
où 
$$
E_k(\beta) = \sum_{i=1}^{n}\e^{z_i\beta}\left[(v_k - v_{k-1})\ind_{x_i\geq v_k} +(x_i - v_{k-1})\ind_{(v_{k-1},v_k]}(x_i)\right],
$$ 
est l'exposition.
\begin{remark}
Dans le papier original de \citet{Breslow1975}, les données censurés entre $v_{k-1}$ et $v_k$ sont remplacées par $v_{k-1}$ (l'observation non censuré la plus proche). Cela simplifie l'exposition qui devient 
$$
E_k(\beta) = \sum_{i=1}^{n}\e^{z_i\beta}(v_k - v_{k-1})\ind_{x_i\geq v_k}.
$$
\end{remark}
On recherche $\beta$ qui annule
$$
\frac{\partial}{\partial \beta}l[\Data; (\widetilde{\alpha},\beta) ] = \sum_{k=1}^K s_k - d(v_k)\frac{E_k'(\beta)}{E_k(\beta)}.
$$
Le programme d'optimisation alterne les mises à jour des $\tilde{\alpha}$ et $\beta$ (numériquement pour $\beta$). Les estimateur de $\alpha$ et $\beta$ sont convergent et asymptotiquement normaux puisque issu du maximum de vraisemblance, la matrice de variance-covariance est donnée par la matrice Hessienne de la log vraisemblance
$$
I_n(\theta) = \left[-\frac{\partial^2}{\partial\theta^2}l\left(\Data;\widehat{\theta}\right)\right].
$$ La fonction de hasard cumulée de base est estimée par 
$$
\widehat{H}_0(t) = \sum_{k=1}^K\frac{d(v_k)}{E_k(\widehat{\beta})}\ind_{v_k\leq t}.
$$
Il est possible d'obtenir une loi asymptotique pour cette estimateur, voir \citet{Spiekerman1998}, afin de construire des intervalles de confiance.
\section{Test d'hypothèse}
\subsection{Significativité des coefficients: le test de Wald}
Pour un paramètre univarié $\theta$ dont l'estimateur $\widehat{\theta}$est asymptotiquement normal. On peut effectuer une test de Wald pour tester sa nullité
$$
(H_0): \theta =0\text{  }(H_1):\theta\neq 0.
$$
La statistique de test est donnée par
$$
\frac{\widehat{\theta}}{\widehat{\text se}}\sim\NormalDist(0,1),
$$ 
où $\widehat{\text se}$ est une estimation de l'écart-type type de $\widehat{\theta}.$ Ici la racine carré du coefficient diagonal correspondant de l'inverse de la matrice de Fisher.
\subsection{Significativité du modèle: le test du rapport de vraisemblance}
Le test du rapport de vraisemblance test des hypothèses du type
$$
(H_0): \theta\in \Theta_0.\text{  }(H_1):\theta \notin\Theta_0,
$$
où $\Theta_0$ est une restriction de l'espace des paramètres $\Theta$, avec $\Theta_0\subset \Theta$. La statistique du test corrrespond est un ratio de log vraisemblance donnée par 
$$
\lambda = 2\log\frac{\underset{\theta \in \Theta}{\sup}\mathcal{L}\left(\Data ; \theta\right)}{\underset{\theta \in \Theta_0}{\sup}\mathcal{L}\left(\Data ; \theta\right)}.
$$
L'ensemble $\Theta_0$ consiste souvent en l'ensemble $\Theta$ pour lequel certain paramètre sont fixés. Supposons que $\theta = (\theta_1,\ldots,\theta_p)$ et que l'on souhaite tester l'hypothèse
$$
(H_0):\theta\in \Theta_0= \{\theta : \theta_1 = \theta_1^0,\ldots,\theta_q = \theta^0_q\},\text{ tel que } q \leq p, 
$$
alors 
$$
\lambda \sim \chi^2_{q}.
$$
La significativité du modèle de Cox test l'hypothèse de nullité simultanée de tout les coefficients. Concrètement 
$$
(H_0): \beta_1=0,\ldots\beta_p = 0\text{ }(H_1):\beta_1\neq0,\ldots\beta_p \neq 0,
$$
et $\lambda \sim \chi_p^2$.
\subsection{Vérification de l'hypothèse de proportionalité}
Dans le cadre du modèle de Cox, on s'intéresse à la loi jointe du couple $(T,Z)$. Supposons que $Z$ soit un vecteur aléatoire sur un espace d'état discret $E$ et $T$ soit également une \va discrète. Le modèle de Cox donne la fonction de hasard de la durée $T$ sachant le vecteur de covariable $Z$, c'est à dire 
$$
\Prob(T = t|Z =z,T\geq t) = h(t|z) = h_0(t)\e^{z\beta}.
$$
La loi de $Z|T=t$ est obtenue par 
\begin{eqnarray*}
\Prob(Z=z|T=t) &=& \frac{\Prob(Z=z,T=t)}{\Prob(T=t)}\\
&=&\frac{\Prob(Z=z,T=t, T\geq t)}{\sum_{z\in E}\Prob(T=t, Z=z, T\geq t)} \\
&=& \frac{\Prob(T=t|Z=z,T\geq t)\Prob(Z=z,T\geq t)}{\sum_{z\in E}\Prob(T=t| Z=z, T\geq t)}\\
&=& \frac{h_0(t)\e^{z\beta}\Prob(Z=z,T\geq t)}{\sum_{z\in E}\Prob(T=t| Z=z, T\geq t)\Prob(Z=z, T\geq t)}\\
&=&\frac{\E(e^{\beta Z}\ind_{\{Z = z, T\geq t\}})}{\E(e^{\beta Z}\ind_{T\geq t})}
\end{eqnarray*}
En présence d'un échantillon \iid $(t_i, z_i)_{i = 1,\ldots, n}$, cette loi peut être estimée par 
$$
\widehat{\Prob}(Z=z|T=t) = \frac{\sum_{i = 1}^n\e^{\widehat{\beta} z_i}\ind_{\{ z_i = z, t_i\geq t\}}}{\sum_{i = 1}^n \e^{\widehat{\beta} z_i}\ind_{t_i\geq t}}.
$$
cette estimateur demeure valide en remplaçant les observation $t_i$ par les observations censurées à droite $x_i$. Soient $v_1<\ldots < v_K$ les temps d'occurence des évènements distincts et non censurés. On peut comparer les caractéristiques moyenne observée et attendu des évènement s'étant produit aux temps $v_k,k=1,\ldots, K$ en calculant la différence
$$
r_k = Z(v_k) - \E(Z|T = v_k) = \frac{\sum_{i=1}^nz_i\ind_{x_i = v_k}}{d(v_k)}  - \sum_{i = 1}^nz_i\frac{\e^{\widehat{\beta} z_i}\ind_{\{ x_i\geq v_k\}}}{\sum_{i = 1}^n \e^{\widehat{\beta} z_i}\ind_{x_i\geq v_k}}.
$$
Les $r_k$ sont appelés résidus de \citet{SCHOENFELD1982}. Nous allons tester l'hypothèse de proportionalité séparément pour chacune des variables explicatives, cela revient à considérer $Z$ et $\beta$ comme des scalaires. On définit une fonction de hasard alternative avec 
$$
h^{\text{alt}}(t|z) = h_0(t)\exp[z\beta(t)] = h_0(t)\exp[z(\widehat{\beta}+\gamma g(t)],
$$
pour $\gamma\in \RL$ et $g:\RL_+\rightarrow\RL$. On test la significativité du coefficient $\gamma$ avec 
$$
(\text{H}_0): \gamma = 0.
$$
Les résidus se ré-écrivent 
$$
r_k = Z(v_k) - \E^\text{alt}(Z|T = v_k) + [\E^\text{alt}(Z|T = v_k)-\E(Z|T = v_k)] = Z(v_k) - M(\widehat{\beta}(v_k), v_k) + [M(\widehat{\beta}(v_k), v_k)-M(\widehat{\beta}, v_k)]
$$
On effectue un développement limité de $M(\widehat{\beta}(v_k), v_k)$  au voisinage de $\widehat{\beta}$. Il vient 
$$
r_k = Z(v_k) - M(\widehat{\beta}(v_k), v_k) + V(\widehat{\beta}, v_k)\gamma g(v_k),
$$
où
$$
V(\widehat{\beta}, v_k) = \frac{\partial}{\partial x} M(x, v_k)\Big|_{x = \widehat{\beta}} \approx \E(Z^2|T = v_k) - \E(Z|T = v_k)^2.
$$
L'espérance sous $H_0$ des résidus standardisés $r_k^\ast  = r_k /V(\widehat{\beta}, v_k)$ vérifie 
$$
\E(r_k^\ast) = \gamma g(v_k).
$$
Cela suggère une relation linéaire entre les $r_k^\ast$ et les $g(v_k)$. On peut choisir pour $g$ une fonction arbitraire. Un choix commun est l'identité. La significativité du coefficient $\gamma$ est établi par un un test de student sur le coefficient $\gamma$ vu comme la pente de la regression linéaire entre les série $r_k^\ast$ et $g(t_k)$. La significativité de toutes les covariables simultanément est possible également, grâce aux travaux de \citet{GRAMBSCH1994}.


