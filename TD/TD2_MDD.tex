\documentclass[11pt, answers]{exam}
\usepackage[utf8]{inputenc}
\usepackage[T1]{fontenc}
\usepackage{amsmath, amssymb, amsopn, color, tikz, mathtools}
\usepackage[margin=1in]{geometry}
\usepackage{titlesec}
\usepackage{tipa}
\usepackage{hyperref}


% Style
\setlength\parindent{0pt}
\shadedsolutions

% Define course info
\def\semester{Semestre 1}
\def\course{Modèle de duréee M1 DUAS}
\def\name{P.-O. Goffard}
%\def\quizdate{10/5, 10/6}
\def\hwknum{}
%\def\title{\MakeUppercase{Homework \hwknum -- quiz \quizdate }}
\def\title{\MakeUppercase{TD 2: Approche non-paramétrique de la survie.}}
% Distributions.
\newcommand*{\UnifDist}{\mathsf{Unif}}
\newcommand*{\ExpDist}{\mathsf{Exp}}
\newcommand*{\DepExpDist}{\mathsf{DepExp}}
\newcommand*{\GammaDist}{\mathsf{Gamma}}
\newcommand*{\LognormalDist}{\mathsf{LogNorm}}
\newcommand*{\WeibullDist}{\mathsf{Weib}}
\newcommand*{\ParetoDist}{\mathsf{Par}}
\newcommand*{\NormalDist}{\mathsf{Normal}}

\newcommand*{\GeometricDist}{\mathsf{Geom}}
\newcommand*{\NegBinomialDist}{\mathsf{NegBin}}
\newcommand*{\BinomialDist}{\mathsf{Bin}}
\newcommand*{\PoissonDist}{\mathsf{Poisson}}

% Sets of numbers.
\newcommand*{\RL}{\mathbb{R}}
\newcommand*{\N}{\mathbb{N}}
\newcommand*{\NZ}{\mathbb{N}_0}
% \newcommand*{\NL}{\mathbb{N}_+}

\newcommand*{\cond}{\mid}
\newcommand*{\given}{\,;\,}

%Probability symbols
\newcommand*{\Prob}{\mathbb{P}}
\newcommand*{\Q}{\mathbb{Q}}
\newcommand*{\E}{\mathbb{E}}
\newcommand*{\F}{\mathcal{F}}
% Regarding spacing and abbreviations.
\usepackage{xspace}

% Acronyms
% \@\xspace doesn't add space if next char is punctuation
% However, these will give 2 .'s if used at end of sentence.
\newcommand*{\eg}{e.g.\@\xspace}
\newcommand*{\ps}{p.s.\@\xspace}
\newcommand*{\ie}{i.e.\@\xspace}
\newcommand*{\va}{v.a.\@\xspace}
\newcommand*{\iid}{i.i.d.\@\xspace}
\newcommand*{\ssi}{s.s.i.\@\xspace}
\newcommand*{\cf}{c.f.\@\xspace}
\newcommand*{\pdf}{p.d.f.\@\xspace}
\newcommand*{\pmf}{p.m.f.\@\xspace}
\newcommand*{\cdf}{c.d.f.\@\xspace}
\newcommand*{\SMC}{\textbf{SMC}\@\xspace}
\newcommand*{\MCMC}{\textbf{MCMC}\@\xspace}
\newcommand*{\VF}{\textbf{VF}\@\xspace}


\newcommand*{\iidSim}{\overset{\mathrm{i.i.d.}}{\sim}}
\newcommand*{\bt}{\bm{\theta}}
\newcommand*{\bTheta}{\bm{\Theta}}
\newcommand*{\bbeta}{\bm{\beta}}
\newcommand*{\bx}{\mathbf{x}}
\newcommand*{\bs}{\bm{s}}
\newcommand*{\bu}{\bm{u}}
\newcommand*{\bn}{\bm{n}}

% Roman versions of things.
\newcommand*{\dd}{\mathop{}\!\mathrm{d}}
\newcommand*{\e}{\mathrm{e}}
\DeclareMathOperator*{\argmax}{arg\,max}
\DeclareMathOperator*{\argmin}{arg\,min}

\newcommand*{\norm}[1]{\lVert{} #1\rVert}

% \DeclarePairedDelimiterXPP{\ind}[1]{\ind_}{\{}{\}}{}{#1}
\newcommand*{\ind}{\mathbb{I}}
\def\euro{\mbox{\raisebox{.25ex}{{\it =}}\hspace{-.5em}{\sf C}}}
  \everymath{\displaystyle}
% \newcommand{\limsup}{\overline{\lim}\,}            % blackboard P
% \newcommand{\liminf}{\underline{\lim}\,}            % blackboard P

\begin{document}

% Heading
{\center \textsc{\Large\title}\\
	\vspace*{1em}
	\course -- \semester\\
	\name\\
	\vspace*{2em}
	\hrule
\vspace*{2em}}

\begin{questions}
\question Soit une variable aléatoire de fonction de hasard constante par morceau avec
$$
h(t) = \sum_{l=1}^k\alpha_l\ind_{[v_{l},v_{l+1})}(t),
$$
où $0=v_1 <v_2\ldots< v_{k+1} = \infty$, et $\alpha_1,\ldots, \alpha_k\geq 0$.
\begin{parts}
	\part Si $k = 1$, quel est la loi de $T$.
	\begin{solution}
	Il s'agit de la loi exponentielle de paramètre $1/\alpha_1$
	\end{solution}
	\part Pour $k> 1$, donner l'expression de la fonction de survie
.
\begin{solution}
On a 
\begin{eqnarray*}
H(t) &=& \int_0^t h(s)\text{d}s\\
&=& \sum_{l=1}^k 0\cdot \alpha_l \cdot\ind_{t<v_l} + \alpha_l\cdot(t-v_{l})\ind_{t\in\left[v_{l}, v_{l+1}\right)}+\alpha_l\cdot(v_{l+1}-v_{l})\ind_{t\geq v_{l+1}}\\
&=&\text{Faire le graphique de }t\mapsto0\cdot \alpha_l \cdot\ind_{t<v_l} + \alpha_l\cdot(t-v_{l})\ind_{t\in\left[v_{l}, v_{l+1}\right)}+\alpha_l\cdot(v_{l+1}-v_{l})\ind_{t\geq v_{l+1}}\\
&=& \sum_{l=1}^k\alpha_l(t\land v_{l+1}-v_{l})_+ = \sum_{l=1}^k\alpha_l\max(t\land v_{l+1} - v_{l}, 0)
\end{eqnarray*}
puis $S(t) = \exp[-H(t)]$.
\end{solution}
% \part Donner l'espérance de la durée résiduelle au dela du seuil $t_0$, définie par 
% $$
% \E(T-t_0|T>t_0),
% $$
% en fonction de $S(t_0)$ et de la fonction de hasard cumulée.
% \begin{solution}
% \begin{eqnarray*}
% \E(T-t_0|T>t_0)&=&\E[(T-t_0)\ind_{T>t_0}]/S(t)\\
% &=& \frac{1}{S(t_0)}\int_{t_0}^\infty (t-t_0) f(t)\text{d}t\\
% &=& \frac{1}{S(t_0)}\int_{t_0}^\infty S(t)\text{d}t\\
% &=& \frac{1}{S(t_0)}\int_{t_0}^\infty \exp[-H(t)]\text{d}t\\
% \end{eqnarray*}
% \end{solution}
\part Soit un échantillon de $n$ observations \iid et censurée à droite (censure non informartive)
$$
\mathcal{D} = (x_i, \delta_i) = (t_i\land c_i, \ind_{t_i\leq c_i}),\text{ }i = 1,\ldots, n.
$$
Donner l'estimateur du maximum de vraisemblance des paramètres $\alpha_l$ pour $l = 1,\ldots, k$. L'estimateur obtenu porte le nom d'estimateur de Hoem \cite{Hoem1971}, il est très populaire en science actuarielle.\\
\underline{Indication:} L'estimateur doit faire apparaitre un nombre d'évènement dans le segment $[v_{l},v_{l+1})$
$$
d(v_l) = \sum_{i=1}^n\delta_i\ind_{[v_{l},v_{l+1})}(x_i)
$$
et une exposition au risque
$$
e(v_l) = \sum_{i=1}^n(x_i\land v_{l+1} - v_{l})_+ =\sum_{i=1}^n\max(x_i\land v_{l+1} - v_{l}, 0),
$$
qui s'interprête ici comme la somme des temps passés par les individus sur le segment $[v_l, v_{l+1})$.
\begin{solution}
La vraisemblance s'écrit 
\begin{eqnarray*}
\mathcal{L}(\mathcal{D};\theta)&=&\prod_{i=1}^nh(x_i)^{\delta_i}S(x_i)\\
&=& \prod_{i=1}^n\prod_{l=1}^k\alpha_l^{\delta_i\ind_{[v_l, v_{l+1})}(x_i)}\exp\left\{ -\alpha_l(x_i\land v_{l+1} - v_{l})_+\right\}\\
&=&\prod_{l=1}^k\alpha_l^{d(v_l)}\exp\left\{ -\alpha_le(v_l)\right\}.\\
\end{eqnarray*}
La log-vraisemblance est donnée par 
$$
l(\mathcal{D};\theta) = \sum_{l=1}^k\ln(\alpha_l)d(v_l)-\alpha_le(v_l)
$$
On en déduit l'estimateur du maximum de vraisemblance en résolvant les équations du score avec 
$$
\widehat{\alpha_l} = \frac{d(v_l)}{e(v_l)}.
$$
\end{solution}
\part Donner une estimation de la matrice d'information de Fisher. En déduire une estimation de la variance asymptotique (valide pour un grand nombre d'observations) et un intervalle de confiance de niveau $\epsilon$ pour les paramètres.
\begin{solution}
La matrice d'information de Fisher est estimée par 
$$
(I_n(\widehat{\theta}))_{i,j} = -\frac{\partial}{\partial\alpha_i\partial\alpha_j}l(\mathcal{D};\theta) = \begin{cases}e(v_i)^2/d(v_i),&\text{ si }i=j,\\
0,&\text{ sinon.}
\end{cases}
$$
On en déduit que 
$$
\mathbb{V}(\widehat{\alpha}_l) = d(v_l)/e(v_l)^2
$$
et 
$$
\alpha_l\in[\widehat{\alpha}_l \pm z_{1-\epsilon/2}\mathbb{V}(\widehat{\alpha}_l)]
$$
\end{solution}
% \part Peut on proposer un test pour l'hypothèse $\alpha_1=\ldots = \alpha_k = \alpha_0$.
% \begin{solution}
% On peut proposer un test de Wald avec comme statistique de test
% $$
% \,^t(\alpha-\alpha_0) \widehat{I}_n^{-1}(\alpha-\alpha_0) \sim\NormalDist(0,\text{Id})
% $$
% \url{https://en.wikipedia.org/wiki/Wald_test}
% \end{solution}
\end{parts}
\question Le modèle de Gompertz-Makeham, voir \cite{Makeham1860}, définie la fonction de hasard d'une v.a $T>0$ par 
$$
h(t) = b\cdot c^t,\text{  pour }t\geq 0.
$$
avec $b,c\geq 0$. il s'agit d'un modèle adapté à la modélisation de la durée de vie humaine, les paramètres $b$ et $c$ accomodent l'augmentation progressive du risque avec l'âge.
\begin{parts}
\part Donner l'expression de la fonction de hasard cumulé de $T$.
\begin{solution}
La fonction de hasard cumulé est donnée par 
$$
H(t) = \int_0^th(s)\text{ds} = \frac{b}{\ln c}\left(c^t - 1\right).
$$

\end{solution}
\part Soit $\theta = (b,c)$ et 
$$
q_\theta(x) = \mathbb{P}(T \leq x+1|T>x)
$$
la probabilité de décès à l'âge $x$. Montrer que les probabilités de décès s'écrivent sous la forme 
$$
q_\theta(x) = 1- f^{c^x(c-1)},
$$
où vous exprimerez et $f$ en fonction de $b$ et $c$.
\begin{solution}
On a 
\begin{eqnarray*}
q_\theta(x) =  &=& 1-\exp\left(-\int_{x}^{x+1}bc^t\text{d}t\right)\\
&=& 1-\exp\left(-H(x+1)+H(x)\right)\\
% &=& 1-\exp\left(-a-b\int_{x}^{x+1}e^{x\log(c)}\text{d}x\right)\\
&=& 1-\exp\left(-\frac{b}{\log(c)}c^x(c-1)\right)\\
\end{eqnarray*}
On identifie $f = e^{-b/\log(c)}$.
\end{solution}
\part Montrer que 
$$
\log(q_\theta(x)) \approx \alpha + \beta x
$$
\underline{Indication}: On pourra utiliser un développement limité.
\begin{solution}
On fait l'approximation 
$1-\exp\left(-\frac{b}{\log(c)}c^x(c-1)\right)\approx \frac{b}{\log(c)}c^x(c-1)$
qui correspond à un développement limité à l'ordre 1 faisant l'hypothèse que $\frac{b}{\log(c)}c^x(c-1)$ est proche de $0$ puis on passe au log pour obtenir
$$
\log(q_\theta(x))=\log(f) + \log(c-1) + \log(c)x
$$
% Pour $q_\theta(x)$ proche de $0$, on a 
% $$
% q_\theta(x)\approx-\log(1-q_\theta(x)) = -\log(d) - \log(f)(c-1)c^x.
% $$ 
\end{solution}
% \part En utilisant la question précédente identifier les coefficients $\alpha$ et $\beta$ tel que
% $$
% \log(q_{\theta}(x+1)-q_{\theta}(x))\approx \alpha +\beta x
% $$
% en fonction de $\theta$. 
% \begin{solution}
% En utilisant l'approximation de la question précédente, il vient 
% \begin{eqnarray*}
% \log(q_{\theta}(x+1)-q_{\theta}(x))&\approx& \log\left[ \log(1/f)(c-1)^2c^x\right]\\
% &=&\log(\log(1/f))+2\log(c-1) + x\log(c)\\
% &=&\alpha +\beta x
% \end{eqnarray*}
% \end{solution}
\part En utilisant le modèle de l'exercice $1$ avec $v_0 = 0, v_1 = 1, v_2 = 2,\ldots$, exprimer 
$$
q_{\text{hoem}}(x) = \mathbb{P}(T \leq x+1| T>x)
$$
en fonction de $\alpha_x$ pour $x=0,1,\ldots$.
\begin{solution}
Il vient $q_{\text{hoem}}(x) = 1 - e^{-\alpha_x}$.
\end{solution}
\part Soit un échantillon de $n$ observations \iid et censurée à droite 
$$
\mathcal{D} = (x_i, \delta_i) = (t_i\land c_i, \ind_{t_i\leq c_i}),\text{ }i = 1,\ldots, n.
$$ Déduire des deux questions précédentes une méthode d'estimation pour $c$ et $f$.
\begin{solution}
On estime $\alpha_x$ par $\hat{\alpha}_x = d(x) / e(x)$, où 
$$
d(x) = \sum_{i=1}^n\delta_i\ind_{[x,x+1)}(x_i)
$$
et 
$$
e(x) = \sum_{i=1}^n(x_i\land (x+1) - x)_+.
$$
Puis on estime $\hat{q}_{\text{hoem}}(x) = 1 - e^{-\hat{\alpha}_x}$. On trouve $\alpha$ et $\beta$ dans 
$$
\log(\hat{q}_{\text{hoem}}(x))\approx \hat{\alpha} +\hat{\beta} x
$$
par les moindre carrés ordinaires puis on trouve $c$, $f$ et $b$.
\end{solution}
\part Ecrire un code R permettant de faire l'inférence du modèle de Gompertz-Makeham. Le code doit comprendre
\begin{enumerate}
\item Ecrire une fonction pour générer des données depuis le modèle de GM. Simuler un échantillon de $1,000$ observations $t_1,\ldots, t_{1000}$ avec
$$
b= 0.0001\text{ et } c= 1.1.
$$
Puis tracer un histogramme.
\item  Vérifier que l'échantilloneur fonctionne bien en comparant les probabilité de décès $q(x)$ théorique et empirique. Les taux de hasards empiriques doivent être estimés de manière non paramétrique. 

\item Ecrire un code qui retourne l'estimateur des paramètres du modèle de Gompertz-Makeham à partir d'un échantillon comprenant des données censurés à droite. Donner la valeur estimée des paramètre sur l'échantillon généré précédemment en ajoutant une censure à droite non informative tel que 
$$
c_1,\ldots, c_{1000}\sim \text{Exp}(1/\bar t).
$$
Vous pouvez utiliser la méthode de l'exercice ou bien utiliser le maximum de vraisemblance. 
\item  Comparer les fonctions de survies théoriques, empirique estimée via Kaplan-Meier et estimé paramétriquement suivant le modèle de Gompertz Makeham.
\end{enumerate}
% \part En supposant connu $c$ et $f$, comment identifier $d$?
% \begin{solution}
% On reprend l'aproximation de la question c) et on écrit
% $$
% q_{\text{hoem}}(x)\approx -\log(d) - \log(f)(c-1)c^x.
% $$
% On en déduit que
% $$
% d= \exp(-q_{\text{hoem}}(x) - c^x(c-1)\log(f)),\text{ pour tout }x.
% $$
% On peut alors estimer $d$ en prenant la moyenne sur tout les $x$.
% \end{solution}
% \part Ecrire un code R permettant de faire l'inférence du modèle de Gompertz-Makeham. Le code doit comprendre
% \begin{enumerate}
% \item Ecrire une fonction pour générer des données depuis le modèle de GM. Simuler un échantillon de $5,000$ observations $t_1,\ldots, t_{5000}$ avec

% $$
% a= 0.001, b= 0.0001\text{ et } c= 1.1.
% $$
% Puis tracer un histogramme.
% \item  Vérifier que l'échantilloneur fonctionne bien en comparant les taux de hasard théorique et empirique. Les taux de hasard empirique doivent être estimé de manière non paramétrique. On définit une grille de points equidistants 
% $$
% v_0<v_1<\ldots <v_k
% $$
% et on compare $d(v_k)/n(v_k)$ à $h(v_k)  (v_{k+1}-v_k)$ via un graphique qui comprend les deux courbes de taux de hasard. (Prenez un écart de $1$ entre les $v_k$).

% \item Ecrire un code qui retourne l'estimateur des paramètres du modèle de Gompertz-Makeham à partir d'un échantillon comprenant des données censurés à droite. Donner la valeur estimée des paramètre sur l'échantillon généré précédemment en ajoutant une censure à droite non informative tel que 
% $$
% c_1,\ldots, c_{5000}\sim \text{Exp}(1/\bar t),
% $$
% % où $\bar{t} = \frac{1}{n}\sum_k t_k$. Prenez comme valeurs initiales dans l'algorithme d'optimisation les valeurs suivantes 
% % $$
% % a =0.01, b = 0.01\text{, et }c = 1.2
% % $$
% \item  Comparer les fonctions de survies théoriques, empirique estimée via Kaplan-Meier et estimé paramétriquement suivant le modèle de Gompertz Makeham.
% \end{enumerate}
% \end{parts}
% % \question Etude de la loi inverse Gaussienne
% % \begin{parts}
% % \part Expression de la fdr
% % \part Expression de la transformée de Laplace

% % \part Expression de l'estimateur par la méthode des moments et distribution de l'estimateur

% % \part Mise en place d'un test d'adéquation de type CvM en présence de données censurées (type III), remplacement de la fdr empirique par KP. Estimation via le maximum de vraisemblance.
\end{parts}
\end{questions}
\bibliography{../lecture_notes/mdd.bib}
\bibliographystyle{plain}
\end{document}
