\documentclass[11pt, answers]{exam}
\usepackage[utf8]{inputenc}
\usepackage[T1]{fontenc}
\usepackage{amsmath, amssymb, amsopn, color, tikz, mathtools}
\usepackage[margin=1in]{geometry}
\usepackage{titlesec}
\usepackage{tipa}
\usepackage{hyperref}


% Style
\setlength\parindent{0pt}
\shadedsolutions

% Define course info
\def\semester{Semestre 2}
\def\course{Modèle de duréee M1 DUAS}
\def\name{P.-O. Goffard}
%\def\quizdate{10/5, 10/6}
\def\hwknum{}
%\def\title{\MakeUppercase{Homework \hwknum -- quiz \quizdate }}
\def\title{\MakeUppercase{TD 3: Modèle à hasard proportionel.}}
% Distributions.
\newcommand*{\UnifDist}{\mathsf{Unif}}
\newcommand*{\ExpDist}{\mathsf{Exp}}
\newcommand*{\DepExpDist}{\mathsf{DepExp}}
\newcommand*{\GammaDist}{\mathsf{Gamma}}
\newcommand*{\LognormalDist}{\mathsf{LogNorm}}
\newcommand*{\WeibullDist}{\mathsf{Weib}}
\newcommand*{\ParetoDist}{\mathsf{Par}}
\newcommand*{\NormalDist}{\mathsf{Normal}}

\newcommand*{\GeometricDist}{\mathsf{Geom}}
\newcommand*{\NegBinomialDist}{\mathsf{NegBin}}
\newcommand*{\BinomialDist}{\mathsf{Bin}}
\newcommand*{\PoissonDist}{\mathsf{Poisson}}

% Sets of numbers.
\newcommand*{\RL}{\mathbb{R}}
\newcommand*{\N}{\mathbb{N}}
\newcommand*{\NZ}{\mathbb{N}_0}
% \newcommand*{\NL}{\mathbb{N}_+}

\newcommand*{\cond}{\mid}
\newcommand*{\given}{\,;\,}

%Probability symbols
\newcommand*{\Prob}{\mathbb{P}}
\newcommand*{\Q}{\mathbb{Q}}
\newcommand*{\E}{\mathbb{E}}
\newcommand*{\F}{\mathcal{F}}
% Regarding spacing and abbreviations.
\usepackage{xspace}

% Acronyms
% \@\xspace doesn't add space if next char is punctuation
% However, these will give 2 .'s if used at end of sentence.
\newcommand*{\eg}{e.g.\@\xspace}
\newcommand*{\ps}{p.s.\@\xspace}
\newcommand*{\ie}{i.e.\@\xspace}
\newcommand*{\va}{v.a.\@\xspace}
\newcommand*{\iid}{i.i.d.\@\xspace}
\newcommand*{\ssi}{s.s.i.\@\xspace}
\newcommand*{\cf}{c.f.\@\xspace}
\newcommand*{\pdf}{p.d.f.\@\xspace}
\newcommand*{\pmf}{p.m.f.\@\xspace}
\newcommand*{\cdf}{c.d.f.\@\xspace}
\newcommand*{\SMC}{\textbf{SMC}\@\xspace}
\newcommand*{\MCMC}{\textbf{MCMC}\@\xspace}
\newcommand*{\VF}{\textbf{VF}\@\xspace}


\newcommand*{\iidSim}{\overset{\mathrm{i.i.d.}}{\sim}}
\newcommand*{\bt}{\bm{\theta}}
\newcommand*{\bTheta}{\bm{\Theta}}
\newcommand*{\bbeta}{\bm{\beta}}
\newcommand*{\bx}{\mathbf{x}}
\newcommand*{\bs}{\bm{s}}
\newcommand*{\bu}{\bm{u}}
\newcommand*{\bn}{\bm{n}}

% Roman versions of things.
\newcommand*{\dd}{\mathop{}\!\mathrm{d}}
\newcommand*{\e}{\mathrm{e}}
\DeclareMathOperator*{\argmax}{arg\,max}
\DeclareMathOperator*{\argmin}{arg\,min}

\newcommand*{\norm}[1]{\lVert{} #1\rVert}

% \DeclarePairedDelimiterXPP{\ind}[1]{\ind_}{\{}{\}}{}{#1}
\newcommand*{\ind}{\mathbb{I}}
\def\euro{\mbox{\raisebox{.25ex}{{\it =}}\hspace{-.5em}{\sf C}}}
  \everymath{\displaystyle}
% \newcommand{\limsup}{\overline{\lim}\,}            % blackboard P
% \newcommand{\liminf}{\underline{\lim}\,}            % blackboard P

\begin{document}

% Heading
{\center \textsc{\Large\title}\\
	\vspace*{1em}
	\course -- \semester\\
	\name\\
	\vspace*{2em}
	\hrule
\vspace*{2em}}

\begin{questions}

\question On considère un modèle de hasard proportionnel pour lequel le risque de base est donnée par une loi de Weibull. La fonction de hasard de $T$ conditionellement au vecteur de covariables $Z$ est donnée par 
$$
h(t):= h(t|z;\alpha, \beta)= \alpha t^{\alpha-1}\e^{z\beta}.
$$
Soit un échantillon de données censurées à droite 
$$
(x_i,\delta_i) = (t_i\land c_i,\ind_{t_i\leq c_i}),\text{ }i=1,\ldots, n.
$$
\begin{parts}
\part Ecrire la log vraisemblance du modèle.
\begin{solution}
La vraisemblance s'écrit
$$
\mathcal{L}(\mathcal{D};\theta) = \prod_{i=1}^n(\alpha x_i^{\alpha-1}\e^{\,^tz_i\beta})^{\delta_i}\exp(\e^{\,^tz_i\beta}x_i^\alpha).
$$
La log vraisemblance est donnée par 
$$
l(\mathcal{D};\theta) = \ln(\alpha)\sum_{i=1}^n\delta_i +(\alpha-1)\sum_{i=1}^n\delta_i\ln(x_i) + \sum_{i=1}^n\delta_i \,^tz_i\beta-\sum_{i=1}^n\e^{\,^tz_i\beta}x_i^\alpha.
$$
\end{solution}
\part Ecrire l'expression des dérivées premières de la log vraisemblance.
\begin{solution}
Les dérivées premières sont données par  
$$
\frac{\partial}{\partial \alpha}l(\mathcal{D};\theta) = \frac{1}{\alpha}\sum_{i=1}^n\delta_i +\sum_{i=1}^n\delta_i\ln(x_i) - \sum_{i=1}^n\e^{\,^tz_i\beta}\ln(x_i)x_i^\alpha.
$$
et 
$$
\frac{\partial}{\partial \beta}l(\mathcal{D};\theta) =  \sum_{i=1}^n\delta_i z_i-\sum_{i=1}^nz_i \e^{\,^tz_i\beta}x_i^\alpha.
$$
\end{solution}
% \part Ecrire l'expression des dérivées secondes de la log vraisemblance. 
% \begin{solution}
% Les dérivées premières sont données par  
% $$
% \frac{\partial^2}{\partial \alpha^2}l(\mathcal{D};\theta) = -\frac{1}{\alpha^2}\sum_{i=1}^n\delta_i - \sum_{i=1}^n\e^{\,^tz_i\beta}\ln(x_i)^2x_i^\alpha,
% $$
% $$
% \frac{\partial^2}{\partial \beta^2}l(\mathcal{D};\theta) =  -\sum_{i=1}^n\,^tz_i z_i \e^{\,^tz_i\beta}x_i^\alpha.
% $$
% et 
% $$
% \frac{\partial^2}{\partial \alpha\partial\beta}l(\mathcal{D};\theta) = - \sum_{i=1}^nz_i\e^{\,^tz_i\beta}\ln(x_i)x_i^\alpha.
% $$
% \end{solution}
\part Ecrire un code R qui simule des données depuis un modèle à hasard proportionel de Weibull et ajuste le modèle aux données. 
\begin{enumerate}
	\item Ecrire une fonction qui prend en entrée $\alpha$, $\beta$ et la matrice des covariables.
	\item Produire un dataframe contenant $1,000$ observations 
	\begin{itemize}
		\item pour chaque individu, on dispose de deux covariables $Z_1\sim \BinomialDist(1, p = 1/2)$ et $Z_2\sim\UnifDist([0,1])$.
		\item Les observations non censurées $t_i$ sont distribués suivant un modèle de Weibull à hasard proportionel tel que $\alpha = 1/2$ et $\beta = (-1/2\text{ }1/2)$
		\item Les observations censurées à droite $x_i$ proviennent d'une variable de censure non informative $c_i$ suivant le même modèle que les observations c'est à dire un modèle de Weibull à hasard proportionel tel que $\alpha = 1/2$ et $\beta = (-1/2\text{ }1/2)$.
	\end{itemize}
	\item Ecrire une fonction pour ajuster le modèle aux données et appliquer cette fonction à votre dataframe.
	\item Vous pouvez comparer le résultat avec celui rendu par la fonction \texttt{survreg}\footnote{Voir \url{http://dwoll.de/rexrepos/posts/survivalParametric.html} pour un tuto} de la librairie \text{survival}.
\end{enumerate}
\end{parts}
\end{questions}
% \bibliography{../lecture_notes/mdd.bib}
% \bibliographystyle{plain}
\end{document}
