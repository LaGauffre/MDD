\documentclass[11pt, answers]{exam}
\usepackage[utf8]{inputenc}
\usepackage[T1]{fontenc}
\usepackage{amsmath, amssymb, amsopn, color, tikz, mathtools}
\usepackage[margin=1in]{geometry}
\usepackage{titlesec}
\usepackage{tipa}
\usepackage{hyperref}


% Style
\setlength\parindent{0pt}
\shadedsolutions

% Define course info
\def\semester{Semestre 2}
\def\course{Modèle de duréee M1 DUAS}
\def\name{P.-O. Goffard}
%\def\quizdate{10/5, 10/6}
\def\hwknum{}
%\def\title{\MakeUppercase{Homework \hwknum -- quiz \quizdate }}
\def\title{\MakeUppercase{Practice Examen}}
% Distributions.
\newcommand*{\UnifDist}{\mathsf{Unif}}
\newcommand*{\ExpDist}{\mathsf{Exp}}
\newcommand*{\DepExpDist}{\mathsf{DepExp}}
\newcommand*{\GammaDist}{\mathsf{Gamma}}
\newcommand*{\LognormalDist}{\mathsf{LogNorm}}
\newcommand*{\WeibullDist}{\mathsf{Weib}}
\newcommand*{\ParetoDist}{\mathsf{Par}}
\newcommand*{\NormalDist}{\mathsf{Normal}}

\newcommand*{\GeometricDist}{\mathsf{Geom}}
\newcommand*{\NegBinomialDist}{\mathsf{NegBin}}
\newcommand*{\BinomialDist}{\mathsf{Bin}}
\newcommand*{\PoissonDist}{\mathsf{Poisson}}

% Sets of numbers.
\newcommand*{\RL}{\mathbb{R}}
\newcommand*{\N}{\mathbb{N}}
\newcommand*{\NZ}{\mathbb{N}_0}
% \newcommand*{\NL}{\mathbb{N}_+}

\newcommand*{\cond}{\mid}
\newcommand*{\given}{\,;\,}

%Probability symbols
\newcommand*{\Prob}{\mathbb{P}}
\newcommand*{\Q}{\mathbb{Q}}
\newcommand*{\E}{\mathbb{E}}
\newcommand*{\F}{\mathcal{F}}
% Regarding spacing and abbreviations.
\usepackage{xspace}

% Acronyms
% \@\xspace doesn't add space if next char is punctuation
% However, these will give 2 .'s if used at end of sentence.
\newcommand*{\eg}{e.g.\@\xspace}
\newcommand*{\ps}{p.s.\@\xspace}
\newcommand*{\ie}{i.e.\@\xspace}
\newcommand*{\va}{v.a.\@\xspace}
\newcommand*{\iid}{i.i.d.\@\xspace}
\newcommand*{\ssi}{s.s.i.\@\xspace}
\newcommand*{\cf}{c.f.\@\xspace}
\newcommand*{\pdf}{p.d.f.\@\xspace}
\newcommand*{\pmf}{p.m.f.\@\xspace}
\newcommand*{\cdf}{c.d.f.\@\xspace}
\newcommand*{\SMC}{\textbf{SMC}\@\xspace}
\newcommand*{\MCMC}{\textbf{MCMC}\@\xspace}
\newcommand*{\VF}{\textbf{VF}\@\xspace}


\newcommand*{\iidSim}{\overset{\mathrm{i.i.d.}}{\sim}}
\newcommand*{\bt}{\bm{\theta}}
\newcommand*{\bTheta}{\bm{\Theta}}
\newcommand*{\bbeta}{\bm{\beta}}
\newcommand*{\bx}{\mathbf{x}}
\newcommand*{\bs}{\bm{s}}
\newcommand*{\bu}{\bm{u}}
\newcommand*{\bn}{\bm{n}}

% Roman versions of things.
\newcommand*{\dd}{\mathop{}\!\mathrm{d}}
\newcommand*{\e}{\mathrm{e}}
\DeclareMathOperator*{\argmax}{arg\,max}
\DeclareMathOperator*{\argmin}{arg\,min}

\newcommand*{\norm}[1]{\lVert{} #1\rVert}

% \DeclarePairedDelimiterXPP{\ind}[1]{\ind_}{\{}{\}}{}{#1}
\newcommand*{\ind}{\mathbb{I}}
\def\euro{\mbox{\raisebox{.25ex}{{\it =}}\hspace{-.5em}{\sf C}}}
  \everymath{\displaystyle}
% \newcommand{\limsup}{\overline{\lim}\,}            % blackboard P
% \newcommand{\liminf}{\underline{\lim}\,}            % blackboard P

\begin{document}

% Heading
{\center \textsc{\Large\title}\\
	\vspace*{1em}
	\course -- \semester\\
	\name\\
	\vspace*{2em}
	\hrule
\vspace*{2em}}

\begin{questions}

\question Soient $U$ et $V$ deux variables aléatoires continues, positives et indépendantes de fonction de hasards respectives $h_U$ et $h_V$. On définit $T=\min(U,V)$. Montrer que la fonction de hasard de $T$ s'écrit 
$$
h_T(t) = h_U(t) + h_V(t),\text{ pour tout }t\geq 0.
$$
\begin{solution}
La fonction de survie de $T$ s'écrit
$$
S_T(t)=\Prob(T>t) = \Prob(\min(U,V)>t) = \Prob(U>t, V>t) = S_U(t)S_V(t).
$$
On en déduit que 
$$
f_T(t) = -S_t'(t) = f_U(t)S_V(t) + S_U(t)f_V(t).
$$
On a alors 
$$
h_T(t)=\frac{f_T(t)}{S_T(t)} = h_U(t) + h_V(t)
$$
\end{solution}
\question Soit $T$ une variable aléatoire discrète sur un espace d'état $E = \{t_1,\ldots, t_n\}$ tels que 
$0=t_0<t_1<t_2<\ldots < t_n = \infty$. On note 
$$
h(t_k) = \Prob(T = t_k|T>t_{k-1})\text{ pour }k\geq 2.
$$
Montrer que la fonction de survie de $T$ peut s'écrire
$$
S(t) = \prod_{k:t_k\leq t}[1-h(t_k)]
$$
\begin{solution}
Pour $t>0$, il existe $k<n$ tel que $t\in[t_k,t_{k+1})$
\begin{eqnarray*}
S(t)&=& \Prob(T>t)\\
&=&\Prob(T>t_k)\\
&=&\Prob(T>t_k|T>t_{k-1})S(t_{k-1})\\
&=&(1-\Prob(T\leq t_k|T>t_{k-1}))S(t_{k-1})\\
&=&(1-h(t_k))S(t_{k-1})\\
&=&(1-h(t_k))(1-h(t_{k-1}))S(t_{k-2})\\
&=&\ldots\\
&=&\prod_{k:t_k<t}(1-h(t_{k}))
\end{eqnarray*}
\end{solution}
\question On modélise la durée de vie humaine par une variable aléatoire $X$ de fonction de hasard
$$
h(x) = a+bc^x,
$$
où $a,b,c>0$. On note $\theta = (a,b,c)$. On note 
$$
q_{x}(\theta) = \Prob(X\leq x+1 |X> x)
$$
la probabilité de décès à l'âge $x$. Montrer que dans le cadre du modèle, on a 
$$
q_{x}(\theta) = 1-sg^{c^x(c-1)}
$$
où on exprimera $s$ et $g$ en fonction de $a, b$ et $c$.
\begin{solution}
On a 
\begin{eqnarray*}
q_x(\theta) &=&\Prob(X\leq x+1 |X> x)\\
&=&1 - \Prob(X> x+1 |X> x)\\
&=&1 - \frac{\Prob(X> x+1 ,X> x)}{\Prob(X>x)}\\
&=&1-\frac{S(x+1)}{S(x)}\\
&=& 1-\exp\left(-\int_{x}^{x+1}a+bc^x\text{d}x\right)\\
&=& 1-\exp\left(-a-b\int_{x}^{x+1}e^{x\log(c)}\text{d}x\right)\\
&=& 1-e^a\exp\left(-\frac{b}{\log(c)}c^x(c-1)\right)\\
\end{eqnarray*}
\end{solution}
\question Soit $T\sim\text{Par}(\theta)$ une variable aléatoire de fonction de survie donnée par 
$$
S(t) = \frac{1}{(1+t)^\theta},\text{ pour }t>0
$$
\begin{parts}
\part Donner la densité et la fonction de hasard de $T$.
\begin{solution}
On a 
$$
f(t) = -S'(t) =  \theta(1+x)^{-\theta-1}
$$
et 
$$
h(t =\frac{f(t)}{S(t)}= \frac{\theta}{1+x}
$$
\end{solution}
\part Calculer l'espérance de vie résiduelle, définie par
$$
e(t) = \E(T-t|T>t).
$$
Pour quelle valeur de $\theta$ cette espérance de vie résiduelle est bien définie.
\begin{solution}
L'espérance de vie résiduelle est définie pour $\theta >1$ et on a 
\begin{eqnarray*}
e(t) &= &\E(T-t|T>t)\\
& =& \frac{\E((T-t)\mathbb{I}_{T>t})}{S(t)} \\
&=& (1+t)^{\theta}\int_{t}^{\infty}(s-t)\theta(1+s)^{-\theta - 1}\text{d}s\\
&=&\frac{1+t}{\theta -1}.
\end{eqnarray*}
\end{solution}
\part Soient $t_1,\ldots, t_n$ un échantillon iid suivant $T$. On suppose que l'échantillon est censuré à droite. Donner une écriture de la vraisemblance du modèle prenant en compte la variable aléatoire de censure, notée C, indépendante de $T$.
\begin{solution}
Soient $c_1,ldots, c_n$ des réalisations iid de $C$. Les données disponible sont 
$$
\mathcal{D} = (x_i,\delta_i)=(t_i\land c_i,\mathbb{I}_{t_i\leq c_i}),
$$
et la vraisemblance du modèle s'écrit 
$$
\mathcal{L}(\mathcal{D}, \theta) = \prod_{i=1}^nh(x_i)^{\delta_i}S(x_i)
$$
\end{solution}
\part Supposons que $c_i = c$ pour tout $i=1,\ldots, n$. Donner l'expression du maximum de vraisemblance
\begin{solution}
On résout 
$$
\frac{\partial }{\partial \theta}\log \mathcal{L}(\mathcal{D}, \theta)=0
$$
et on obtient 
$$
\widehat{\theta} = \frac{r}{\sum_{i}x_i}.
$$
où $r=\sum_{i=1}^n\delta_i$. On vérifie que 
$$
\frac{\partial^2 }{\partial \theta^2}\log \mathcal{L}(\mathcal{D}, \widehat{\theta})= -\frac{r}{\widehat{\theta}^2}.
$$
\end{solution}
\part Donner un intervalle de confiance pour $\widehat{\theta}$.
\begin{solution}
On a 
$$
\hat{\theta}\sim\NormalDist\left[\theta, -\left(\frac{\partial^2 }{\partial \theta^2}\log \mathcal{L}(\mathcal{D}, \theta)\right)^{-1}\right]\text{, pour }n\rightarrow \infty
$$
\end{solution}
\part Supposons que la variable aléatoire de censure vérifie $C\sim\text{Par}(\beta \theta)$ pour $\beta>0$. Calculer 
$$
\Prob(T>C),
$$
qui correspond à la probabilité qu'une observation soit censuré.
\begin{solution}
$$\frac{\beta}{\beta+1}$$
\end{solution}
\part Ecrire la vraisemblance du modèle en suppsosant le paramètre $\beta$ inconnu. Donner l'estimater du maximum de vraisemblance de $\theta$ et $\beta$.
\begin{solution}
La vraisemblance s'écrit 
$$
\mathcal{L}(\mathcal{D}, \theta, \beta) = \prod_{i=1}^n(f(x_i;\theta)^{\delta_i}S(x_i;\beta\theta))^{\delta_i}(f(x_i;\beta\theta)^{\delta_i}S(x_i;\theta))^{1-\delta_i}.
$$
On résout les équations du score pour obtenir
$$
\widehat{\beta} = \frac{n}{r}-1\text{, et }\widehat{\theta}=\frac{r}{\sum_i\log(1+x_i)}.
$$
\end{solution}
\end{parts}
\end{questions}
% \bibliography{../lecture_notes/mdd.bib}
% \bibliographystyle{plain}
\end{document}
