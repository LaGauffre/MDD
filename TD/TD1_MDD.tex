\documentclass[11pt, answers]{exam}
\usepackage[utf8]{inputenc}
\usepackage[T1]{fontenc}
\usepackage{amsmath, amssymb, amsopn, color, tikz, mathtools}
\usepackage[margin=1in]{geometry}
\usepackage{titlesec}
\usepackage{tipa}
\usepackage{hyperref}


% Style
\setlength\parindent{0pt}
\shadedsolutions

% Define course info
\def\semester{Semestre 2}
\def\course{Modèle de duréee M1 DUAS}
\def\name{P.-O. Goffard}
%\def\quizdate{10/5, 10/6}
\def\hwknum{}
%\def\title{\MakeUppercase{Homework \hwknum -- quiz \quizdate }}
\def\title{\MakeUppercase{TD 1: Approche paramétrique de la survie.}}
% Distributions.
\newcommand*{\UnifDist}{\mathsf{Unif}}
\newcommand*{\ExpDist}{\mathsf{Exp}}
\newcommand*{\DepExpDist}{\mathsf{DepExp}}
\newcommand*{\GammaDist}{\mathsf{Gamma}}
\newcommand*{\LognormalDist}{\mathsf{LogNorm}}
\newcommand*{\WeibullDist}{\mathsf{Weib}}
\newcommand*{\ParetoDist}{\mathsf{Par}}
\newcommand*{\NormalDist}{\mathsf{Normal}}

\newcommand*{\GeometricDist}{\mathsf{Geom}}
\newcommand*{\NegBinomialDist}{\mathsf{NegBin}}
\newcommand*{\BinomialDist}{\mathsf{Bin}}
\newcommand*{\PoissonDist}{\mathsf{Poisson}}

% Sets of numbers.
\newcommand*{\RL}{\mathbb{R}}
\newcommand*{\N}{\mathbb{N}}
\newcommand*{\NZ}{\mathbb{N}_0}
% \newcommand*{\NL}{\mathbb{N}_+}

\newcommand*{\cond}{\mid}
\newcommand*{\given}{\,;\,}

%Probability symbols
\newcommand*{\Prob}{\mathbb{P}}
\newcommand*{\Q}{\mathbb{Q}}
\newcommand*{\E}{\mathbb{E}}
\newcommand*{\F}{\mathcal{F}}
% Regarding spacing and abbreviations.
\usepackage{xspace}

% Acronyms
% \@\xspace doesn't add space if next char is punctuation
% However, these will give 2 .'s if used at end of sentence.
\newcommand*{\eg}{e.g.\@\xspace}
\newcommand*{\ps}{p.s.\@\xspace}
\newcommand*{\ie}{i.e.\@\xspace}
\newcommand*{\va}{v.a.\@\xspace}
\newcommand*{\iid}{i.i.d.\@\xspace}
\newcommand*{\ssi}{s.s.i.\@\xspace}
\newcommand*{\cf}{c.f.\@\xspace}
\newcommand*{\pdf}{p.d.f.\@\xspace}
\newcommand*{\pmf}{p.m.f.\@\xspace}
\newcommand*{\cdf}{c.d.f.\@\xspace}
\newcommand*{\SMC}{\textbf{SMC}\@\xspace}
\newcommand*{\MCMC}{\textbf{MCMC}\@\xspace}
\newcommand*{\VF}{\textbf{VF}\@\xspace}


\newcommand*{\iidSim}{\overset{\mathrm{i.i.d.}}{\sim}}
\newcommand*{\bt}{\bm{\theta}}
\newcommand*{\bTheta}{\bm{\Theta}}
\newcommand*{\bbeta}{\bm{\beta}}
\newcommand*{\bx}{\mathbf{x}}
\newcommand*{\bs}{\bm{s}}
\newcommand*{\bu}{\bm{u}}
\newcommand*{\bn}{\bm{n}}

% Roman versions of things.
\newcommand*{\dd}{\mathop{}\!\mathrm{d}}
\newcommand*{\e}{\mathrm{e}}
\DeclareMathOperator*{\argmax}{arg\,max}
\DeclareMathOperator*{\argmin}{arg\,min}

\newcommand*{\norm}[1]{\lVert{} #1\rVert}

% \DeclarePairedDelimiterXPP{\ind}[1]{\ind_}{\{}{\}}{}{#1}
\newcommand*{\ind}{\mathbb{I}}
\def\euro{\mbox{\raisebox{.25ex}{{\it =}}\hspace{-.5em}{\sf C}}}
  \everymath{\displaystyle}
% \newcommand{\limsup}{\overline{\lim}\,}            % blackboard P
% \newcommand{\liminf}{\underline{\lim}\,}            % blackboard P

\begin{document}

% Heading
{\center \textsc{\Large\title}\\
	\vspace*{1em}
	\course -- \semester\\
	\name\\
	\vspace*{2em}
	\hrule
\vspace*{2em}}

\begin{questions}
\question L'objectif est de déterminer l'estimateur du maximum de vraisemblance du maximum de vraisemblance de la loi exponentielle $\ExpDist(\beta)$ de densité 
$$
f(t) = \frac{\e^{-t/\beta}}{\beta}\ind_{(0,\infty)}(t).
$$
Soit $\mathcal{D} = \{t_1, \ldots,t_n\}$ un échantillon \iid de réalisation d'une \va exponentielle.
\begin{parts}
\part Donner l'estimateur du maximum de vraisemblance $\widehat{\theta}$.
\begin{solution}
La log vraisemblance s'écrit
$$
l(\mathcal{D};\theta) = -n\log(\beta)-\frac{1}{\beta}\sum t_i.
$$
On a 
$$
\frac{\partial l}{\partial \theta}(\mathcal{D};\theta) = -n\frac{n}{\beta}+\frac{1}{\beta^2}\sum t_i.
$$
On en déduit que 
$$
\frac{\partial l}{\partial \theta}(\mathcal{D};\theta) = 0\Leftrightarrow \widehat{\beta} = \frac{\sum t_i}{n},
$$
après avoir vérifié que
$$
\frac{\partial l}{\partial \theta}(\mathcal{D};\widehat{\theta}) < 0.
$$
\end{solution}
\part Supposons que nous soyons en présence de données censurées, avec une censure à droite, non informative, de niveau $c_1,\ldots, c_n$. Donner l'estimateur du maximum de vraisemblance de $\beta$.

\begin{solution}
Les observations sont 
$$
(x_i,\delta_i) = (t_i\land c_i, \mathbb{I}_{t_i\leq c_i}),
$$
et la log vraisemblance s'écrit
$$
-\sum \delta_i\log(\beta)-\frac{\sum x_i}{\beta} + \text{Cste}.
$$
On en déduit que
$$
\frac{\partial l}{\partial \theta}(\mathcal{D};\theta) = 0\Leftrightarrow \widehat{\beta} = \frac{\sum x_i}{\sum \delta_i}.
$$
On peut vérifier la cohérence de l'estimation. En prenant $C\rightarrow \infty$, on retombe sur l'estimateur obtenu à la question 1.
\end{solution}

% \part Supposons que nous soyons en présence de données censurées, avec une censure de type II avec arrêt au $r^{\text{ème}}$ évènement et $r\leq n$. Donner l'estimateur du maximum de vraisemblance de $\beta$.
% \begin{solution}
% En cas de censure de type II, les observations sont 
% $$
% (x_i,\delta_i) = (t_i\land t_{(r)}, \mathbb{I}_{t_i\leq t_{(r)}}),
% $$
% L'estimateur est exactement le même qu'à la question 2.
% \end{solution}

% \part Supposons que nous soyons en présence de données censurées, avec une censure de type de niveau $c_1,\ldots, c_n$. Donner l'estimateur du maximum de vraisemblance de $\beta$.
% \begin{solution}
% En cas de censure de type III, les observations sont 
% $$
% (x_i,\delta_i) = (t_i\land c_i, \mathbb{I}_{t_i\leq c_i}),
% $$
% L'estimateur est exactement le même qu'à la question 2.
% \end{solution}

\end{parts}
\question Nous allons construire un test d'adéquation à la loi exponentielle $\ExpDist(\beta)$ de densité 
$$
f(t) = \frac{\e^{-t/\beta}}{\beta}\ind_{(0,\infty)}(t),
$$
basé sur la transformée de Laplace. La transformée de Laplace d'une \va $T$ est donnée par 
$$
\psi(\theta) = \mathbb{E}(\e^{-\theta T}).
$$
\begin{parts}
\part Soit $t_1, \ldots, t_n$ un échantillon \iid de réalisation de $T\sim \ExpDist(\beta)$. L'estimateur du maximum de vraisemblance de $\beta$ est donnée par 
$$
\widehat{\beta}_n = \frac{1}{n}\sum_{i=1}^n t_i.
$$
Quelle est la distribution de $T/\widehat{\beta}_n$ lorsque $n\rightarrow\infty$?
\begin{solution}
$T/\widehat{\beta}_n\overset{D}{\rightarrow} \ExpDist(1)$
\end{solution}
\part Donner l'expression de la transformée de Laplace $\psi(\theta)$ de $Y\sim \ExpDist(\beta=1)$ et montrer qu'elle vérifie l'équation différentielle suivante
$$
\psi'(\theta)(1+\theta) + \psi(\theta) = 0
$$
\begin{solution}
On a
$$
\psi(\theta) = \frac{1}{1+\theta}
$$
et 
$$
\psi'(\theta) = -\frac{1}{(1+\theta)^2} =- \frac{1}{1+\theta}\psi(\theta)
$$
\end{solution}
\part On définit $y_i = t_i/\widehat{\beta}_n$ et
$$
\psi_n(\theta)=\frac{1}{n}\sum_{i=1}^ne^{-\theta y_i}
$$
A quoi correspond cette quantité?
\begin{solution}
Il s'agit d'un estimateur assymptotiquement sans biais de la transformée de Laplace d'une loi $\ExpDist(1)$.
\end{solution}
\part On définit la statistique de test suivante 
$$
S_n=n\int_{0}^\infty[\psi'_n(\theta)(1+\theta) + \psi_n(\theta)]^2\e^{-\theta}\text{d}\theta,
$$
Quel est votre interprétation de cette statistique de test?
\begin{solution}
On retrouve dans le carré sous l'intégrale, l'équation différentielle satisfaite par $\psi$ dans le cas où les données suivent une loi exponentielle. On intègre pour prendre en compte l'ensemble du domaine de définition de la transformée de Laplace. On ajoute la fonction exponentielle pour rendre la statistique de test intégrable. L'avantage est d'obtenir une test de test explicite, voir la question d'après.
\end{solution}
\part Montrer que l'on peut estimer $S_n$ par 
$$
S_n = \frac{1}{n}\sum_{j,k=1}^n\left[\frac{(1-y_j)(1-y_k)}{y_j+y_k+1}-\frac{y_j + y_k}{(y_j+y_k+1)^2}+\frac{2y_jy_k}{(y_j+y_k+1)^2}+\frac{2y_jy_k}{(y_j+y_k+1)^3}\right].
$$
\\
\underline{Indication:} Il faut développer le carré sous l'intégrale et calculer chaque terme séparément. C'est un peu fastidieux mais on y arrive. 
\begin{solution}
Il s'agit d'un petit calcul intégral, je vous fait confiance.
\end{solution}
\part Illustrer ce test avec R (comparer son efficacité à celui de Kolmogorov-Smirnov).

On fixe le niveau du test à $0.05$. Calculer la probabilité de rejeter l'hypothèse 
$$
(H_0): T\sim\ExpDist(\beta)
$$
lorsque les données sont des réalisations \iid de loi $\GammaDist(\alpha, 1)$. Il s'agit de la puissance du test. Evaluer la puissance du test pour $\alpha = 1/4, 1/2, 3/4, 1, 5/4, 3/2, 7/4.$ et $n=50$. On fera le graphique de la puissance en fonction de $\alpha$ avec une courbe pour le test présenté dans l'exercice (baptisé LT test) et le test de Kolmogorov.\\

\underline{Indications:} Voici les étapes de l'algorithme
\begin{enumerate}
	\item Simuler $t_i\sim\GammaDist(\alpha, 1)$ pour $i=1,\ldots, n$
	\item Estimer $\beta$ par $\widehat{\beta}$
	\item Définir $y_i = t_i/\widehat{\beta}$ et calculer $S_n$
	\item Simuler $\tilde{t}_i \sim\ExpDist(\widehat{\beta})$
	\item Estimer $\beta$ par $\tilde{\beta}$
	\item Définir $\tilde{y}_i = \tilde{t}_i/\tilde{\beta}$ et calculer $\tilde{S}_n$ sur la base des $\tilde{y}_i$
\end{enumerate}
Répéter ces étapes $J = 1000$ fois. On a une suite de valeur de test statistiques $S_n^j$  et $\tilde{S}_n^{j}$ pour $j=1,\ldots, J$. La valeur critique du test est donnée par 
$$
S_{0.95} = \text{Quantile}(\tilde{S}_n^{j}\text{, }j=1,\ldots, J;0,95),
$$
qui correspond au quantile empirique d'ordre $95\%$ des $\tilde{S}_n^{j}$. La puissance du test est alors 
$$
\frac{1}{J}\sum_{j= 1}^{J}\mathbb{I}_{S_n^{j}>S_{0.95}}.
$$
Vous devriez obtenir un résultat proche de celui de la Figure \ref{fig:power_LT_VS_KS}.
\begin{figure}[h!]
\centering
\includegraphics[width = 0.7\textwidth]{../figures/power_LT_VS_KS}
\caption{Puissance du LT test et du test de K-S.}
\label{fig:power_LT_VS_KS}
\end{figure}
\part Commenter le résulat de la Figure \ref{fig:power_LT_VS_KS}.
\begin{itemize}
 \item La forme de la courbe est elle celle attendue? 
 \item Quel est le meilleur test?
\end{itemize}
Pour plus d'information sur ce test, on pourra se référer au travail de Henze et Meintanis \cite{Henze2002}
\begin{solution}
\begin{itemize}
	\item La forme de la courbe est celle attendue, pour $\alpha = 1$ les données provienne du modèle exponentielle, la puissance des test doit donc atteindre le niveau fixé à $0.05$ et augmenté de part et d'autre de $\alpha = 1$
	\item Le LT test semble supérieur. Il est tout à fait possible qu'un résultat inverse soit obtenu si les données proviennent d'un modèle de Weibull par exemple. On peut conclure que le LT test semble meilleur pour discriminer des données qui proviennent de la loi gamma.
\end{itemize}
\end{solution}
\end{parts}

\end{questions}
\bibliography{mdd.bib}
\bibliographystyle{plain}
\end{document}
